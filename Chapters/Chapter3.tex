% Chapter Template

\chapter{Planificació} % Main chapter title

\label{Chapter3} % Change X to a consecutive number; for referencing this chapter elsewhere, use \ref{ChapterX}

En la primera fase del projecte es va realitzar una planificació inicial a on s'especificava quines tasques en realitzarien durant el Treball Final de Grau, com s'organitzarien i, sobretot, quant temps es dedicaria en cada una d'elles. Després d'haver transcorregut tres mesos dels quatre que aproximadament dura el projecte, hi ha hagut punts que es van estimar correctament i s'han pogut dur a terme sense cap tipus de problemàtica però d'altres que s'han vist forçats a modificar. Un cop analitzats es veurà si finalment, fent balanç, ha modificat l'objectiu final de projecte.

\section{Acord amb la planificació inicial}

En el seu moment es va definir una metodologia de treball inspirada en la metodologia àgil Scrum, amb un seguit de cinc sprints o iteracions a on es desenvoluparia la implementació del projecte. L'objectiu era definir un \textit{backlog} inicial amb totes les \textit{històries d'usuari} que contindria el projecte i, en cada un dels \textit{sprint plannings}, atribuir un conjunt d'aquestes a aquell sprint.
\\\\
Després d'aquests tres mesos de projecte es pot dir que s'ha seguit aquest patró o metodologia de treball. A més a més, comentar que la durada de les dues setmanes per iteració que es va atribuir a la planificació inicial s'ha respectat amb una variació màxima de 2 o 3 dies.


\section{Modificacions}

El canvi més important que s'ha realitzat acord amb la planificació inicial del projecte, però que no modifica el fet d'arribar a l'objectiu final en el temps acordat, és l'alteració en l'ordre de les tasques. En el seu moment es va estipular la idea de realitzar durant dos mesos tota la implementació del projecte i, en l'últim més, redactar la memòria documentant tota la implementació realitzada durant la fase intermèdia.
\\\\
El canvi ha estat degut per un seguit de factors. Tant el director del projecte com l'alumne que el porta a càrrec es van adonar que realitzar l'ordre de tasques estipulat inicialment portaria a la problemàtica de no tenir una bona documentació final a entregar. En primer lloc pel poc temps que es disposava en comparació del que es necessita per realitzar una bona memòria, i per altra banda el fet de documentar tot al final podria donar lloc a no realitzar una documentació de la implantació del tot acurada donat el temps entre finalitzar el codi i explicar-lo. A més a més, el fet d'anar intercalant implementació amb documentació faria més amè el transcurs del projecte.
\\\\
Per altra banda, hi ha hagut modificacions en la implementació de les \textit{històries d'usuari}. Inicialment, en realitzar i definir \textit{backlog} inicial, es van crear un seguit de \textit{features} sense contemplar del tot el temps que es disposava per implementar-les. Llavors, durant la realització de la iteració, l'alumne es va adonar que aquestes històries d'usuari aportaven un esforç extra en l'sprint que no es veia reflectit en el resultat final. És a dir, l'esforç que es necessitava per implementar aquestes històries d'usuari no era equivalent al valor que aportava a l'usuari final del producte. En conseqüència, per tal de solucionar aquesta problemàtica, es va decidir presidir d'aquestes \textit{features} donat el temps acotat del projecte.


\section{Conclusions}

Finalment, es pot considerar que tot i les modificacions en comparació de la planificació inicial, no hi haurà cap problemàtica en arribar de forma correcta al \textit{deadline} del projecte. Actualment ja s'ha implementat la primera fase del treball, corresponent a tota història d'usuari referent a la gestió del restaurant, la qual representa més del 60\% del treball.
\\\\
Paral·lelament ja s'ha documentat bona part de la memòria, concretament s'han redactat de forma definitiva els tres primers capítols corresponents a la contextualització, motivació, estat de l'art, objectius, abast, obstacles i metodologia. I a més s'ha començat a redactar altres punts com els agents implicats, tecnologies utilitzades, planificació, gestió econòmica i sostenibilitat.